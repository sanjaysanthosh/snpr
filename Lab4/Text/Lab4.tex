%----------------------------------------------------------------------------------------
%	PACKAGES AND OTHER DOCUMENT CONFIGURATIONS
%----------------------------------------------------------------------------------------

\documentclass[11pt]{article} % Default font size is 12pt, it can be changed here

\usepackage[a4paper]{geometry} % Required to change the page size to A4


\usepackage[pdftex]{graphicx}
\usepackage{listings}    
\lstset{language=C}
\usepackage{xcolor}
\usepackage{float} % Allows putting an [H] in \begin{figure} to specify the exact location of the figure
\usepackage{wrapfig} % Allows in-line images such as the example fish picture
\usepackage{amsmath}
\usepackage{hyperref}
\usepackage{cleveref}
\usepackage{microtype}
\microtypesetup{protrusion=true} % enables protrusion
\linespread{1} % Line spacing

\setlength\parindent{0pt} % Uncomment to remove all indentation from paragraphs



\graphicspath{{Pictures/}} % Specifies the directory where pictures are stored

\lstset{
	language=C,    
  	breaklines=true,%default: false
    basicstyle=\fontsize{10}{10}\ttfamily, %\ttfamily\scriptsize,
    keywordstyle=\color{red}\sffamily\bfseries,                                                                
    commentstyle=\itshape\color{purple!40!black}, 
    identifierstyle=\color{blue},                                 
    %stringstyle=\rmfamily,     
    stringstyle=\color{blue!20!black!10!green!10},                                                 
    showstringspaces=false,%default: true
    backgroundcolor=\color{white},
  	breakatwhitespace=false,
 	breaklines=true,
 	escapeinside={\%*}{*},
  	%columns=fullflexible,
  	xleftmargin=3.2pt,
  	frame=none,%default: none 
    framerule=0.2pt,%expands outward 
    rulecolor=\color{red},
    framesep=3pt,%expands outward
    %numbers=left,
}

\begin{document}

%----------------------------------------------------------------------------------------
%	TITLE PAGE
%----------------------------------------------------------------------------------------

\begin{titlepage}
\begin{center}


\newcommand{\HRule}{\rule{\linewidth}{0.5mm}} % Defines a new command for the horizontal lines, change thickness here


\vspace{30 mm}

\textsc{\large Sensor Networks Lab (PR WS14/15) }\\[1cm] % Name of your university/college

%\HRule \\[0.4cm]
{\huge Lab 4: Advanced UDP communication} \\[1cm] % Title of your document

\begin{minipage}{0.5\textwidth}
\begin{flushleft}
\center
Sanjeet Raj \textsc{Pandey} [313631]\\
Sanjay Santhosh \textsc{Kumar}\\
Muhammad Tauseef \textsc{Khan} [346301]\\
Rajibul Alam \textsc{Khan}\\[2cm]

Supervisor: \\
Vlado \textsc{Handziski}\\ 
Tim \textsc{Bormann} \\
\vspace{30 mm}
\begin{figure}[H]
 \centering
 \includegraphics[width=2cm]{logo}
\end{figure}
Telecommunication Networks Group\\
Technical University Berlin\\ \begin{picture}(2,2)
\put(1,1){\circle*{1}}
\end{picture}
\end{flushleft}

\end{minipage} \\[1cm]
\end{center}

\vspace{30 mm}
\end{titlepage}

\section*{Excercise 4.1: \textnormal{\large{\textit{Echo and integers} Repeat the instructions from the beginning
of the chapter and again try to enter a word and numbers. You will see that
the numbers will not be displayed properly, because netcat will interpret them as
characters, not as numbers.}}}

Entered digit is sent as ASCII code and in big endian order e.g 6 is sent as 360a
while returning it via ECHO code as integer 06000000 shows in wireshark. 
While after converting digit from host to network order digit is sent as 00000006.
One can see how endiness gets swapped while sending integers not to forget UDP network protocol is based on big-endian.


\section*{Excercise 4.3: \textnormal{\large{\textit{Memory Layout} Give an example for a 1-byte, 2-byte and 4-byte
alignment. Use the following sequence: 16 bit, 16 bit, 8 bit, 32 bit and 16 bit.}}}

\begin{figure}[h]
\setlength{\unitlength}{0.115in} % selecting unit length
\centering % used for centering Figure
\begin{picture}(40,20) % picture environment with the size (dimensions)
 % 32 length units wide, and 15 units high.
\put(4.5,19.2) {16 bit}
%\put(4,19){\color{red}\line(1,0){4}}

\put(8.7,19.2) {16 bit}
%\put(8,19){\color{green}\line(1.2,0){4}}

\put(13,19.2) {8 bit}
%\put(12,19){\color{black}\line(1,0){2}}

\put(17.5,19.2) {32 bit}
%\put(14,19){\color{gray}\line(1,0){8}}

\put(25,19.2) {16 bit}
%\put(22,19){\color{blue}\line(1,0){4}}

\put(4,16){\colorbox{red!20}{\framebox(2,2){8}\framebox(2,2){8}}\colorbox{green!20}{\framebox(2,2){8}\framebox(2,2){8}}\colorbox{yellow!20}{\framebox(2,2){8}}\colorbox{blue!20}{\framebox(2,2){8}\framebox(2,2){8}\framebox(2,2){8}\framebox(2,2){8}}\colorbox{gray!20}{\framebox(2,2){8}\framebox(2,2){8}}}

\put(0,14.3){1 Byte Al.}
\put(4,13.5){\colorbox{red!20}{\framebox(2,2){8}\framebox(2,2){8}}\colorbox{green!20}{\framebox(2,2){8}\framebox(2,2){8}}\colorbox{yellow!20}{\framebox(2,2){8}}\colorbox{blue!20}{\framebox(2,2){8}\framebox(2,2){8}\framebox(2,2){8}\framebox(2,2){8}}\colorbox{gray!20}{\framebox(2,2){8}\framebox(2,2){8}}}

\put(0,11.8){2 Byte Al.}
\put(4,11.0){\colorbox{red!20}{\framebox(2,2){8}\framebox(2,2){8}}\colorbox{green!20}{\framebox(2,2){8}\framebox(2,2){8}}\colorbox{yellow!20}{\framebox(2,2){8}\framebox(2,2){8}}\colorbox{blue!20}{\framebox(2,2){8}\framebox(2,2){8}\framebox(2,2){8}\framebox(2,2){8}}\colorbox{gray!20}{\framebox(2,2){8}\framebox(2,2){8}}}

\put(0,8.9){4 Byte Al.}
\put(4,8.5){\colorbox{red!20}{\framebox(2,2){8}\framebox(2,2){8}\framebox(2,2){8}\framebox(2,2){8}}\colorbox{green!20}{\framebox(2,2){8}\framebox(2,2){8}\framebox(2,2){8}\framebox(2,2){8}}\colorbox{yellow!20}{\framebox(2,2){8}\framebox(2,2){8}\framebox(2,2){8}\framebox(2,2){8}}\colorbox{blue!20}{\framebox(2,2){8}\framebox(2,2){8}\framebox(2,2){8}\framebox(2,2){8}}\colorbox{gray!20}{\framebox(2,2){8}\framebox(2,2){8}\framebox(2,2){8}\framebox(2,2){8}}}

\end{picture}
\caption{Alignment in General way for 16,16,8,32,16 bits arrangement}
\end{figure}

Memory alignment must always be taken care of during lower level programming. Depending on the architectures its can have different byte alignment system as in our example we are checkign for 1 byte, 2 byte and 4 byte alignment. 
As for 1 byte alignment every 8 bit is read one after another. But for 2 byte alignment third block with 8 bit makes reading next byte unaligned so padding of 1 byte is added depending on big endian after 8 bit cell or for little endian before 8 bit cell. 
In 4 byte alignment 1st two 16 bit cell and last 16 bit cell gets padded with 2 more byte as well as 8 bit cell on third gets padded with 3 more byte also respect to endianess of system. 


\end{document}